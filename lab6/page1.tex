\documentclass[a4paper, 9pt]{article}
\usepackage[T2A]{fontenc}
\usepackage[utf8]{inputenc}
\usepackage[english,russian]{babel}
\usepackage{indentfirst}
\usepackage{physics}
\usepackage{amsmath}
\usepackage{tikz}
\usepackage{mathdots}
\usepackage{yhmath}
\usepackage{cancel}
\usepackage{color}
\usepackage{siunitx}
\usepackage{array}
\usepackage{multirow}
\usepackage{amssymb} 
\usepackage{gensymb}
\usepackage{tabularx}
\usepackage{extarrows}
\usepackage{booktabs}
\usepackage[letterspace=150]{microtype}
\usepackage{multicol}
\usepackage[inner=0.3in,outer=0.3in,tmargin=2cm,footskip=.1in]{geometry}
\setlength{\columnsep}{0.1cm}

\usepackage{titleps}
\newpagestyle{main}{
	\sethead{}{\lsstyle ФИЗИЧЕСКИЙ ФАКУЛЬТАТИВ}{\textbf{\emph{35}}}
}
\pagestyle{main}  

\begin{document}
    \begin{multicols}{2}
        []
        можно получить
        \begin{equation} 
            - \frac{\Delta N}{\Delta r} = N \frac{mg}{kT}.
        \end{equation}
        
        Из выражений (3) и (4) найдем окончательно
        \begin{equation*}
            -\frac{\Delta n}{n \Delta r} \approx \alpha N \frac{mg}{kT}
        \end{equation*}

        Подставляя сюда значения нужных величин для обеих планет, 
        получим следующую таблицу (здесь $m_p = 1,67 \dot 10^{-27}$ кг - масса протона):
        
        \begin{tabular}{ | m{1cm} | m{1cm} | m{1cm} | m{1cm} | m{1cm} | m{1cm} | m{1cm} | } 
            \hline
             & l & c & d & e & f & g \\
            \hline
        \end{tabular}

        \columnbreak
        Из выражений (3) и (4) найдем окончательно
        \begin{equation*}
            -\frac{\Delta n}{n \Delta r} \approx \alpha N \frac{mg}{kT}
        \end{equation*}
    \end{multicols}
\end{document}